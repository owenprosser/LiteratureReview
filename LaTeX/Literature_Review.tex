\documentclass[conference, a4paper]{IEEEtran}
%\IEEEoverridecommandlockouts
% The preceding line is only needed to identify funding in the first footnote. If that is unneeded, please comment it out.
\usepackage{cite}
\usepackage{amsmath,amssymb,amsfonts}
\usepackage{algorithmic}
\usepackage{graphicx}
\graphicspath{{images/}}
\usepackage{textcomp}
\usepackage{xcolor}
\usepackage{hyperref}
\usepackage{fancyhdr}
\usepackage{filecontents}
\renewcommand{\headrulewidth}{0pt}
\pagestyle{fancy}
\lfoot{Does a higher quality of living contribute to a lower suicide rate?}
\cfoot{}
\rfoot{\thepage}
\usepackage[belowskip=-15pt,aboveskip=0pt]{caption}
\hypersetup{
    colorlinks=false,
    pdfborder={0 0 0},
}

\def\BibTeX{{\rm B\kern-.05em{\sc i\kern-.025em b}\kern-.08em
    T\kern-.1667em\lower.7ex\hbox{E}\kern-.125emX}}
\begin{document}

\title{Detecting DoS Attacks (Working Title)\\
}

\author{\IEEEauthorblockN{Owen Prosser}
\IEEEauthorblockA{\textit{School of Computer Science} \\
\textit{University of Lincoln}\\
Lincoln, United Kingdom \\
14514822@students.lincoln.ac.uk}
}

\maketitle

\begin{abstract}
Write Abstract
\end{abstract}

\begin{IEEEkeywords}
Machine Learning, Cyber Security, Packet Analysis
\end{IEEEkeywords}

Information Security and Network Resilience has become increasingly important as society has become reliant on technology to function.
One example of this is the effect that cyber attacks can have on infrastructure such as Power Systems[1]. One way which networks can be disrupted is a technique known as a Denial of Service attack (DoS).

\begin{thebibliography}{00}
\bibitem{Chen, W., Ding, D., Dong, H. and Wei, G., 2019. Distributed Resilient Filtering for Power Systems Subject to Denial-of-Service Attacks. IEEE Transactions on Systems, Man, and Cybernetics: Systems.}
\bibitem{nomaly-based_method_for_DDoS}Karimazad, R. and Faraahi, A., 2011, September. An anomaly-based method for DDoS attacks detection using RBF neural networks. In Proceedings of the International Conference on Network and Electronics Engineering (Vol. 11, pp. 44-48).
\end{thebibliography}
\vspace{12pt}

\end{document}